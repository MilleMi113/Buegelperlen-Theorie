\documentclass[11pt,a4paper,twoside]{scrreprt}
\usepackage[utf8]{inputenc}
\usepackage[T1]{fontenc}
\usepackage[ngerman]{babel}
\usepackage{graphicx}
\usepackage{wrapfig}
\usepackage{url}
\usepackage{textcomp}
\usepackage{ulem}
\usepackage{comment}
\usepackage[backend=biber,style=alphabetic,]{biblatex}

\title{Kreativitätsspielzeug: rasterbasierte Spielzeuge zur Bilderzeugung und deren Hackability}
\author{Camilla Schmider}

\begin{document}
\maketitle
\tableofcontents

\chapter{Einleitung}

% Einleitung ist noch unvollständig, enthält bisher nur einen sehr groben Überblick über die geplanten Teile der Arbeit

Die Gesamtarbeit gliedert sich in drei Teile: einen mit designgeschichtlichem Schwerpunkt, einen mit designtheoretischem Schwerpunkt und einen mit produktsprachlichem Schwerpunkt. 

Der Designgeschichtliche Schwerpunkt versucht Lego und Bügelperlen als Konstruktions- und Kreativspielzeuge in den Kreativitätsdiskurs ihrer Entstehungszeit einzuordnen.

Der Designtheoretische Schwerpunkt versucht Andreas Reckwitz' Konzept vom Neuen in die Kreativitätsentwicklung beim Kinderspiel zu übertragen/auf selbiges anzuwenden.

Der Produktsprachliche Schwerpunkt versucht die Gestaltungsmerkmale zu identifizieren, die einen Kreativen Umgang mit den Produkten befördern.

\begin{comment}
Für die Einleitung:
-Forschungsgegenstand klar benennen
-Gegenstand einordnen und abgrenzen?
-übergreifende Forschungsfrage(n) für die drei Teile klar formulieren
-Arbeitshypothese(n) formulieren?
-Fachtermini, die über alle Teile hinweg genutzt werden grob anreißen?
\end{comment}

% ab hier: alte Einleitung von 2022/23
In dieser Hausarbeit schreibe ich über rasterbasierte Spielzeuge zur Bilderzeugung und deren Grenzen und Möglichkeiten im Bilderzeugungsprozess. 

Zudem möchte ich untersuchen, welche Produktmerkmale einen Anreiz setzen, das Raster zu durchbrechen und die Elemente anders einzusetzen, als (in der jeweiligen Basisausführung des Spielzeugs) offfensichtlich vorgesehen.

(absatz zur Pracktik des Hackens?)

Als ausgebildete Jugend- und Heimerzieherin und angehende Designerin liegt einer meiner praktischen Arbeitsschwerpunkte in der Gestaltung von Lehr- und Lernmaterialien. Diese Arbeit soll einen kleinen theoretischen Beitrag dazu liefern, wie Spielzeuge gestaltet werden können um Lernanreize zu setzen. Bevorzugt sogar auf eine Weise, dass diese Lernanreize nicht der vordergründige, hauptsächliche oder einzige Zweck ist, sondern das (zweckfreie) Spielen im Vordergrund steht. Es soll also nicht darum gehen, wie Lernspielzeuge gestaltet werden, sondern darum wie mit Spielzeugen das Lernen gestaltet werden kann. 

(in welchem Bereich wird geforscht?)

Unter Rasterbasierten Spielzeugen zur Bilderzeugung verstehe ich Produktsysteme, bei denen auf Grundplatten mit Rasterartig angeordneten Erhöhungen oder Vertiefungen werden einzelne Bildpunkte angeordnet, die in ihrer Gesamtheit ein flächiges Bild erzeugen. Die Bildpunkte können auf unzählige Arten immer wieder neu kombiniert werden, was diese zu Spielzeugen macht und von klassischem Bastelmaterial unterscheidet.

Mein Ausgangspunkt waren Bügelperlen, die aus verschiedensten Sichtweisen heraus als Kreativmaterial und und Spielzeug verstanden werden können. Die deutsche und englische Wikipedia listen sie in den beiden Kategorien "Kunststoffspielzeug" und "Handarbeit" \cite{wiki:de}, bzw. "Plastic toys" und "Craft materials" \cite{wiki:en}, aber auch gängige Definitionen eines Baukastens treffen. \cite{Leinweber}

Bügelperlen machen kreatives Gestalten mit Farbe zum Gegenstand des Spiels, laden ein, sich mit der Ästhetik der erzeugten Bilder zu beschäftigen und bieten Anreize zu kreativer Problemlösung in der Auseinandersetzung mit den Regeln des Produktsystems. \begin{comment} ist das so? Wie kann ich diese These belegen? \end{comment}
Weiterhin können sie selbst als Material zur kreativen Problemlösung im Spiel außerhalb der beschäftigung mit dem Produktsystem eingesetzt werden. (Beispiel 1: Im Spiel mit der Puppe stellt ein Kind fest, dass die Puppe dringend eine Perlenkette braucht. Da keine in geeigneter Größe vorhanden ist, fädelt das Kind schnell ein paar Bügelperlen auf einen Wollfaden und verknotet diesen zu einer kleinen Kette für die Puppe. Beispiel 2: Ein Kind spielt mit einer Holzeisenbahn und baut um diese herum phantasievolle  Landschaften aus Bauklötzen. in die Mitte dieser Eisenbahnlandschaft soll noch ein Ententeich, was mit den vorhandenen Bauklötzen nicht so überzeugend gelingen will. Spontan entscheidet sich das Kind, blaue und transparente Bügelperlen in eine flache Schale zu schütten und das als Teich zu interpretieren. Mit dem Ergebnis nur halb zufrieden, beschließt ein befreundetes Kind, verschiedene blaue Perlen wellenförmig auf einer runden Steckplatte anzuordnen, und mit einem beigen Rand zu versehen, bügelt das Ergebnis und überlässt diesen so entstandenen Teich dem anderen Kind für seine Landschaft.) \begin{comment} Gibt es knappere oder andere Formen der Erläuterung, was ich meine, außer Beispiele auszuformulieren? \end{comment}

Ergänzend möchte ich die Klemmbausteinserie Lego\texttrademark Dots daneben stellen. Als Spielzeug ist es vergleichbar, da es ebenfalls einSystem ist, bei dem Bildpunkte in einem Raster angeordnet werden. Beides sind populäre Spielzeugsysteme mit einer über 50 jährigen Geschichte, mit einer breiten Produktpalette und vielen Sets und Erweiterungen die sich in der jeweiligen Handhabung und ihren Gestaltungsmöglichkeiten unterscheiden können.
\sout{Im Hauptteil werde ich auf diese Unterschiede detailliert eingehen und gehe der Frage nach, welchen Einfluss diese Unterscheidungsmerkmale auf die Spielmöglichkeiten haben.}
%notiz: Bügelperlen sind flach angeordnet, weil sie gebügelt werden sollen. Essentielle Banalität. :D

Die Raster, die den beiden Spielzeugen zugrunde liegen sind nicht-sprachlich-formulierte Regeln. Hier zunächst als untergrund zur Befestigung und als Gestaltungshilfe im Sinne der Rhythmisierung und Strukturierung. (siehe auch \cite{formfindung})
Diese REgeln können aber auch als Herausforderung betrachtet werden: Als Anreiz, Ergebnissoffen zu probieren, was innerhalb oder trotz der Regeln machbar ist. Als Test, wie Belastbar die Regeln sind. Oder um ein ganz spezifisches Ziel zu erreichen, dass den Regeln widerspricht. 

(Kreativitätsverständnis im Zusammenhang mit Bügelperlen)


\iffalse
% alte Gliederung. Inhalte sind nicht komplett hinfällig, müssen aber neu sortiert und nochmal überarbeitet werden.

\chapter{Begriffsklärung und Forschungsstand}

	\section{Spielzeug}
	Was ist Spielzeug? Anwendung und Stellenwert von Regeln im Zusammenhang mit Spielzeug
	
	\section{Forschungsstand zu den untersuchten Spielzeugsystemen und in dieser Arbeit genutzte Begriffe rund um diese}
		\subsection{Bügelperlen und anderes rasterbasiertes Kreativspielzeug in der Forschung}
		ein scheinbar unerforschtes Gebiet.
		\subsection{Klemmbausteine in Forschung und Fan-Fachkultur}
		Ausstellungskatalog `Baukästen', Lego-Fanblogs?

	\section{Kreativitätsbegriff und Kreativitätsdiskurs}
		\subsection{Kreativität nach Reckwitz}
		u.A. Reckwitz ``Design im Kreativdispositiv''
		\subsection{Historische Einordnung der Entstehungsgeschichte von Bügelperlen und Klemmbausteinen}
		u.a. Mareis ``Entwerfen mit System''
		\subsection{pädagogischer Exkurs: Kreativitätsentwicklung im (Kinder-) Spiel}
		u.A. Winnicott "vom Spiel zur Kreativität"



\chapter{Erforschen durch Selbstbeobachtung und Vergleich}
	\section{Methodenwahl}
	Warum Selbstbeobachtung? Wo sind die Fallstricke? Warum der Vergleich? 
		\subsection{Vergleich zwischen Spielzeugsystemen}
		Wie unterscheiden sich LegoDots von Bügelperlen in Spielmöglichkeiten, Gestaltung und Impulsen zum Regelbruch?
		\subsection{Vergleich zwischen Regelwerken/Konventionen und dem realen (kreativen) Spiel}
		um einen Regelbruch festzustellen, müssen erst Regeln und Konventionen definiert werden.
		\subsection{Vergleich zwischen dem spielerischen Umgang mit den Spielzeugsystemen und dem Kreativitätskonzept nach Reckwitz}
		\sout{eventuell auch im vagen Vergleich zum üblichen Umgang mit Stift und Karopapier oder einem vermeintlich völlig unkreativen Spielzeug. (welches könnte das sein?)}
	\section{Aufbau der Selbstbeobachtung}
Wie habe ich die Selbstbeobachtung strukturiert, die Beobachtungssituation gestaltet und alles dokumentiert?


\chapter{Spielregeln und Regelbrüche, sowie deren unterstützende Gestaltungsmerkmale}
Hier fasse ich meine Beobachtungen zusammen: Im Umgang mit den Produkten und durch die Selbstbeobachtung identifizierte Regeln und Konventionen, wahrgenommene Impulse und Versuche, die festgestellten Regeln zu erweitern, umgehen oder brechen.


\chapter{\sout{Verbreitung und Vielfalt von Regelbrüchen in Faninhalten auf populären Social-Media-Plattformen}}
In einem weiteren Schritt versuche ich über eine kurze Analyse einer kleinen Handvoll der populärsten und neuesten Fan-Inhalte herauszufinden, ob sich ähnliche beobachtungen auch bei mir fremden Personen machen lassen, die nicht durch diese Forschungsarbeit beeinflusst sind, sich allerdings offenbar regelmäßiger und/oder längerfristig intensiver hobbymäßig oder professionalisiert mit den jeweiligen Spielzeugsystemen auseinandersetzen.


\chapter{Fazit}
gibt es einen sweetspot zwischen vorgaben und freiheiten, wenn es um kreative nutzung geht, und wo liegt er mutmaßlich, wenn ich die beiden Spielzeugsysteme betrachte? Diese Arbeit kann nicht beantworten, ob die hier getroffenen Annahmen oder gewonnenen Erkenntnisse für verschiedene Altersklassen und Entwicklungsstufen zutreffen oder gültig sind. Das überschreitet sowohl den Rahmen der Arbeit und meine Fachkompetenz im Feld der Entwicklungspsychologie.

\chapter{Anhang}
Protokolle der Selbstbeobachtung? 

\fi
\chapter{Literaturverzeichnis/Quellen}
%nochmal nachgucken, was von beidem denn jetzt aktuell dem "guten Stil" entspricht.
\printbibliography

\iffalse

\chapter{\LaTeX{}\\Cheatsheet}
Dieses Kapitel wird am Ende wieder rausgelöscht, enthält einfach einige nützliche Copy+Paste-Vorlagen.
\section{Sonderzeichen}
\textcopyright
\textregistered
\texttrademark

\section{durchstreichung}
Zunächst muss das ulem-Paket \begin{code}\usepackage{ulem}\end{code} eingebunden werden, anschließend können Zeichen oder Zeilen mit \begin{quotation}\sout{text}\end{quotation} durchgestrichen werden.


\section{Listen}

Wie funktionieren Listen in \LaTeX{}?\\

\begin{itemize}
\item erster Stichpunkt 
\item zweiter Stichpunkt
\item dritter Stichpunkt
\end{itemize}

Alternativ gibt es auch nummerierte Listen:

\begin{enumerate}
\item Hier steht dann zuerst eine 1 
\item Hier erwartungsgem\"a\ss{} eine 2
\item Und hier die 3
\end{enumerate}


Diese nummerierten Listen lassen sich auch relativ leicht
mit Buchstaben anstelle von Zahlen durchnummerieren. 
Dazu wird die Ausgabe einfach auf Buchstaben umgestellt:


\renewcommand{\labelenumi}{\alph{enumi}}
\begin{enumerate}
\item Stichpunkt 1
\item Stichpunkt 2
\item Stichpunkt 3
\end{enumerate}

Oder auch die Beschreibung:

\begin{description}
\item[Stichpunkt 1]{ Stichpunkt 1 handelt von \dots}
\item[Stichpunkt 2]{ Stichpunkt 2 handelt von \dots}
\item[Stichpunkt 3]{ Stichpunkt 3 handelt von \dots}
\end{description}

\fi

\end{document}
